\subsection*{Human Subjects}

Data used in this study were drawn from the ongoing longitudinal RESONANCE study
of healthy and neurotypical brain and cognitive development,
based at Brown University in Providence, RI, USA.
RESONANCE is part of the NIH initiative Environmental influences on Child Health Outcomes (ECHO) \cite{Forrest2018-ud,Gillman2018-om},
a longitudinal observational study of healthy and neurotypical brain development
that spans the fetal and infant to adolescent life stages,
combining neuroimaging (magnetic resonance imaging, MRI), neurocognitive assessments, bio-specimen analyses, subject genetics,
environmental exposures such as lead, and rich demographic, socioeconomic, family and medical history information.
From the RESONANCE cohort, 344 typically-developing children
between the ages of 28 days and 15 years old were selected for analysis in this study. 

General participant demographics are provided in Tables \ref{tab:demographics} and \ref{tab:agestats} and Figure \ref{fig:data}.
Complete metadata are available in Data Records (see below), with children being representative of the RI population.
As a broad background, children in the RESONANCE cohort were born full-term (>37 weeks gestation)
with height and weight normal for gestational age, and from uncomplicated singleton pregnancies.
Children with known major risk factors for developmental abnormalities at enrollment were excluded.
In addition to screening at the time of enrollment,
on-going screening for worrisome behaviors using validated tools was performed
to identify at-risk children and remove them from subsequent analysis.

Exclusion criteria included: \emph{in utero} exposure to alcohol, cigarette or illicit substance exposure;
preterm (<37 wks gestation) birth; small for gestational age or less than 1500 g; fetal ultrasound abnormalities;
preeclampsia, high blood pressure, or gestational diabetes; 5 minute APGAR scores <8;
NICU admission; neurological disorder (e.g., head injury resulting in loss of consciousness, epilepsy);
and psychiatric or learning disorder (including maternal depression) in the infant, parents, or siblings requiring medication in the year prior to pregnancy.

Demographic and other non-biospecimen data such as race and ethnicity, parental education and occupation,
feeding behavior (breast- and formula-feeding), child weight and height,
were collected through questionnaires or direct examination as appropriate.
All data were collected at every assessment visit.
All procedures for this study were approved by the local institutional review board at Rhode Island Hospital,
and all experiments adhered to the regulation of the review board.
Written informed consent was obtained from all parents or legal guardians of enrolled participants.


\subsection*{Cognitive Assessments}

Overall cognitive function was assessed using age-appropriate methods.
For children from birth to 30 months, we used an Early Learning Composite
as assessed via the Mullen Scales of Early Learning (MSEL) \cite{Mullen1995-ty},
a standardized and population-normed tool for assessing fine and gross motor,
expressive and receptive language, and visual reception functioning in children from birth through 68 months of age.

The third edition of the Bayley Scales of Infant and Toddler Development \cite{Bayley2006-wm}
is a standard series of measures used primarily to assess the development of infants and toddlers,
ranging from 1 to 42 months of age.

The Wechsler Intelligence Quotient for Children (WISC) \cite{Wechsler2012-mi}
is an individually administered standard intelligence test for children aged 6 to 16 years.
It derives a full scale intelligence quotient (IQ) score, which we used to assess overall cognitive functioning.
The fourth edition of the Wechsler Preschool and Primary Scale of Intelligence (WPPSI-IV) \cite{Wechsler2012-mi}
is an individually administered standard intelligence test for children aged 2 years 6 months to 7 years 7 months,
trying to meet the increasing need for the assessment of preschoolers.
Just as the WISC, it derives a full scale IQ score, which we used to assess overall cognitive functioning.


\subsection*{Stool Sample Collection and Sequencing}

Stool samples (n=493) were collected by parents in OMR-200 tubes (OMNIgene GUT, DNA Genotek, Ottawa, Ontario, Canada),
immediately stored on ice, and brought within 24 hrs to the lab in RI where they were immediately frozen at -80 $^{\circ}$C.
Stool samples were not collected if the subject had taken antibiotics within the last two weeks.
DNA extraction was performed at Wellesley College (Wellesley, MA).
Nucleic acids were extracted from stool samples using the RNeasy PowerMicrobiome kit
automated on the QIAcube (Qiagen, Germantown, MD), excluding the DNA degradation steps.
Extracted DNA was sequenced at the Integrated Microbiome Resource (IMR, Dalhousie University, NS, Canada)

Shotgun metagenomic sequencing was performed on all samples.
A pooled library (max 96 samples per run) was prepared using the Illumina Nextera Flex Kit for MiSeq and NextSeq from 1 ng of each sample.
Samples were then pooled onto a plate and sequenced
on the Illumina NextSeq 550 platform using 150+150 bp paired-end “high output” chemistry,
generating ~400 million raw reads and ~120 Gb of sequence per plate.

\hl{Are we still planning to include 16S data here?}

For sequencing 16S rRNA gene amplicons,
the V4-V5 region of the 16S ribosomal RNA gene was sequenced according to the protocol
described by Comeau et al. \cite{Comeau2017-jg}.
Briefly, the V4-V5 region was amplified once using the Phusion High-Fidelity DNA polymerase
(ThermoFisher Scientific, Waltham, MA) and universal bacterial primers
515F: 5’-GTGYCAGCMGCCGCGGTAA-3’ and 926R: 5’-CCGYCAATTYMTTTRAGTTT-3’ \cite{Parada2016-uz,Walters2016-fi}.
These primers had appropriate Illumina adapters and error-correcting barcodes unique to each sample
to allow up to 380 samples to be simultaneously run per single flow cell.
After being pooled into a single library and quantified fluorometrically,
samples were cleaned-up and normalized using the high-throughput Charm Biotech Just-a-Plate 96-well Normalization Kit (Charm Biotech, Cape Girardeau, MO).
The normalized samples were sequenced on the Illumina MiSeq platform (Illumina, San Diego, CA)
using 300+300 bp paired-end V3 chemistry, producing ~55,000 raw reads per sample.

\subsection*{Computational Analysis}

Shotgun metagenomic sequences were analyzed using the bioBakery suite of computational tools \cite{McIver2018-yc}.
First, \verb|KneadData| (v0.7.7) was used to perform quality control of raw sequence reads,
such as read trimming and removal of reads matching a human genome reference.
Next, \verb|MetaPhlAn| (v3.0.7, using database \verb|mpa_v30_CHOCOPhlAn_201901|) was used to generate taxonomic profiles
by aligning reads to a reference database of marker genes.
Finally, \verb|HUMAnN| (v3.0.0a4) was used to functionally profile the metagenomes.

Raw amplicon sequences were profiled using Quantitative Insights in Microbial Ecology 2 (QIIME2) v2021.2.0 \cite{Bolyen2019-qq}.
Briefly, primers flanking V4-V5 were removed from fastq reads using the cutadapt (v3.2) QIIME2 plugin \cite{Martin2011-zv}.
Raw sequence reads of samples from enrichment cultures were denoised, filtered and clustered into amplicon sequence variants (ASVs) using the Divisive Amplicon Denoising Algorithm (DADA2) plugin in QIIME 2 \cite{Callahan2016-ol}.
After denoising and filtering, 16270 total sequences were recovered with a mean length of 373 bases (270-465, standard deviation 13.21).
Taxonomy was assigned to each ASV using a Naïve-Bayes classifier compared against SILVA v.138 reference database \cite{Yilmaz2013-rj,Quast2013-hc}
trained on the 515F-806R region of the 16S rRNA gene \cite{Bokulich2018-dv}.

Additional data processing, generation of summary statistics, 
and generation of plots was performed using the julia programming language \cite{Bezanson2017-ud}.
See Code Availability section for additional details.