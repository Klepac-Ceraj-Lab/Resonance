Both the brain and microbiome of humans develop rapidly in the first years of life,
enabling extensive signaling between the gut and central nervous system (dubbed the “microbiome-gut-brain axis”).
Emerging evidence implicates gut microorganisms and microbiota composition in cognitive outcomes
and neurodevelopmental disorders (e.g., autism),
but the influence of gut microbial metabolism on typical neurodevelopment has not been explored in detail.
We investigated the relationship of the microbiome with the neuroanatomy and cognitive function
of \hl{XXX} healthy children,
demonstrating that differences in gut microbial taxa and gene functions
are associated with the size of brain regions and with overall cognitive function.
Many species, including \hl{XXX} and \hl{XXX}, were associated with higher cognitive function, 
while some species such as \hl{XXX} was more commonly found in children with low cognitive scores.
Microbial enzymes involved in the metabolism of neuroactive compounds such as glutamate and GABA,
were also associated with structure of the brain \hl{eg XXX and YYY} and with overall cognitive function.