\subsubsection{Cohort overview and summary data}

We investigated the co-development of the brain and the microbiome in
early childhood in a cohort of typically-developing children in their
first years of life using a variety of orthogonal microbial and
neurocognitive assessments, including shotgun metagenomic sequencing,
cognitive and behavioral assessments, and neuroimaging. In all, 361
children from RESONANCE,
an accelerated longitudinal cohort of early development
\cite{forrestAdvancingScienceChildren2018}
between the ages of 82 days and 10 years
old (median age 2.21 years, Supplementary Figure 1) were included in
this study (Table 1, Figure 1A). To measure cognitive function, we used
age-appropriate assessments that can be normalized to a common, IQ-like
scale (Figure 1B), including the Mullen scale of early learning (MSEL)
for children from birth to 3 years of age
\cite{mullenMullenScalesEarly1995},
Wechsler preschool and primary scale of intelligence (WPPSI)
for 4-5 year-olds
\cite{wechslerWechslerPreschoolPrimary2012},
and the Wechsler intelligence scale for children (WISC)
for children 6 years and up
\cite{wechslerWechslerIntelligenceScale1949}.

As expected, the greatest differences in microbial taxa were driven by
age (Figure 1B-C, PCoA axis 1), while older children were primarily
stratified into Bacteroidetes-dominant, Firmicutes-dominant, or
high-\emph{Prevotella copri} (Supplementary Figure 2). Overall variation
in gut microbial genes and also in brain volume profiles was similarly
driven largely by subject age (Figure 1C-F).

\includegraphics[width=6.45833in,height=3.87221in]{media/image3.png}

\textbf{Figure 1: Cohort overview and summary data.}

(A) Study design and overview. Stool samples, cognitive assessments and
neuroimaging were collected from participants in an accelerated
longitudinal design. (B) Cognitive function scores are assessed by
different instruments, but may be normalized to a common IQ-like scale.
(C-D) Principal coordinates analysis using Bray-Curtis dissimilarity on
taxonomic profiles demonstrates high beta diversity, with much of the
first axis of variation explained by increasing age and alpha diversity.
Differences in gene function profiles (E) and neuroimaging (PCA based on
the Euclidean distance of brain region volumes) (F) are likewise
dominated by changes as children age. (G) Permutation analysis of
variance (PERMANOVA) of taxonomic profiles, functional profiles
(annotated as UniRef90s) and neuroimaging, against the metadata of
interest. Variations in taxonomic and functional profiles explain a
modest but significant percent of the variation in cognitive development
in children over 18 months of age. (H) Mantel testing of different
microbial feature matrices, shows overlapping but distinct patterns of
variation. Dotted lines in (B) and (C) show 6 and 18 months, which are
used as cut-offs in some following models.

\textbf{Table1: Subject demographics}


Several studies have demonstrated links between specific taxa and
measures of anxiety and depression
\cite{mayneris-perxachsMicrobiotaAlterationsProline2022,needhamGutderivedMetaboliteAlters2022},
cognitive flexibility in adults
\cite{magnussonRelationshipsDietrelatedChanges2015},
and atypical neural development
\cite{liuAlteredGutMicrobiota2019,wanUnderdevelopmentGutMicrobiota2021}.
We, therefore, set out to identify whether specific taxa
or gene functions were linked with normal cognitive development in
children. In order to assess whether variations in gut microbial taxa,
their genes, or their metabolism are linked with neurocognitive
development, we tested whether the beta diversity of microbial taxa and
gene functions, as well as variation in brain development as assessed by
neuroimaging, were associated with these contemporaneous measures of
cognitive function using permutation analysis of variance (PERMANOVA).
Due to the large ecological shift in the microbiome that occurs upon the
introduction of solid food, and a relatively wide range of ages when
infants are transitioned to solid food, we considered the pre-transition
(less than 6 months old) and post-transition (over 18 months old)
microbiomes separately. Even still, age was a major driver of variation
in gut microbiomes in children over 18 months old (Figure 1G). We also
found that overall variation in microbial species in children over 18
months old was associated with a small but significant variation in
cognitive function score (Figure 1G, R\textsuperscript{2} = 0.0124, q
\textless{} 0.001), as was variation in microbial gene functions
(R\textsuperscript{2} = 0.0119, q \textless{} 0.001). Variation in
microbial taxa and genes was not significantly associated with cognitive
function in children under 6 months, though this may be due to the low
taxonomic diversity, and broad lack of overlap between taxa in infants.
As expected, age was significantly associated with microbial beta
diversity (taxa R\textsuperscript{2} = 0.0215, and UniRef90s
R\textsuperscript{2} = 0.0245, q \textless{} 0.001), and very strongly
associated with neuroimaging profiles (R\textsuperscript{2} = 0.256, q
\textless{} 0.001).

Consistent with prior studies, different microbial measurement types
captured overlapping variation, with species profiles and gene function
profiles (annotated with clusters of 90\% similarity
\cite{suzekUniRefComprehensiveNonredundant2007}
), both generated from DNA sequencing data, being tightly coupled
(Figure 1H, p \textless{} 0.001). Interestingly, other functional
groupings (Enzyme commission level4 - ECs
\cite{bairochENZYMEDatabase20002000},
and
KEGG orthologues - KOs
\cite{kanehisaKEGGResourceDeciphering2004}
) overlapped only slightly with taxonomic profiles in both age
cohorts, despite being derived from UniRef90 labels. In children over 18
months, some variation (15.9\%, p \textless{} 0.01) in neuroimaging
overlapped with microbial measures, though this may be due to the
residual variation due to age in both measures.

\subsubsection{Microbial species and neuroactive genes are associated with cognitive performance}

To assess whether individual microbial species were associated with
cognitive function, we fit multivariable linear regression
\cite{mallickMultivariableAssociationDiscovery2021}
to the relative abundance of each species that had at least 15\%
prevalence in a given age group (Figure 2A, N = 92 for 0--120 months, N
= 46 for 0--6 months, N = 97 for 18--120 months). Only \emph{Blautia
wexlerae} was significantly associated (q value = 0.14, β = 0.0015) with
cognitive function in children under 6 months old after adjusting for
age and maternal education (Figure 2B). \emph{B. wexlerae} was
previously shown to be depleted in children with diabetes
\cite{benitez-paezDepletionBlautiaSpecies2020},
and that oral administration of \emph{B. wexlerae} partially
ameliorated weight gain and inflammation from a high-fat diet in a mouse
model of T2D
\cite{hosomiOralAdministrationBlautia2022,liuBlautiaNewFunctional2021}.
In children over 18 months of age,
several microbial species were significantly enriched (q-value
\textless{} 0.20) in children with higher cognitive function scores,
including \emph{Gordonibacter pamelaeae}, which produces the
neuroprotective metabolite urolithin
\cite{gongUrolithinAlleviatesBloodbrain2022,selmaDescriptionUrolithinProduction2014},
\emph{Asaccharobacter celatus} and
\emph{Adelcreutzia equolifaciens}, which produce phytoestrogen-derived
equol
\cite{maruoAdlercreutziaEquolifaciensGen2008,thawornkunoBiotransformationDaidzeinEquol2009},
and the SCFA-producing probiotic
species such as \emph{Eubacterium eligens} and \emph{Faecalibacterium
prausnitzii}
\cite{ghoshMediterraneanDietIntervention2020}
(Figure
2B).

\includegraphics[width=6.5in,height=4.83333in]{media/image4.png}

\textbf{Figure 2 - Taxa and gene functional groups are associated with cognitive function.}

(A) Volcano plot of multivariable linear model results showing the
relationship between individual taxa and cognitive function score in
children over 18 months of age, controlling for age, maternal education,
and sequencing depth. Taxa that were significant after FDR correction (q
\textless{} 0.2) are colored red. (B) For taxa that were significantly
associated with cognitive function, heatmaps of prevalence, mean
relative abundance, and correlation with cognitive function in different
age groups. (C-F) Enrichment plots for selected neuroactive gene sets
used in feature set enrichment analysis (FSEA). (G) summary of FSEA
results across all samples (left) as well as in the under 6 months
(middle) and over 18 months (right) subsets, colored based on the
significance of the association. Markers indicate the individual
correlation of genes within a gene set, and vertical bars represent the
median correlation of that gene set.

Given that different microbial species might occupy the same metabolic
niche in different individuals, we hypothesized that microbial genes
grouped by functional activity would be associated with cognition. To
test this, we performed feature set enrichment analysis (FSEA) on groups
of genes with neuroactive potential
\cite{valles-colomerNeuroactivePotentialHuman2019}
and concurrent cognitive function score (Table 2,
\protect\hyperlink{_i7fd3z19jdvg}{\uline{Figure 2}}C-G) and found that
several metabolic pathways were significantly enriched or depleted in
children with higher cognitive function scores. This was true both when
considering all age groups together, though the enrichment of most
pathways was more pronounced in children under 6 months or over 18
months. For example, genes for degrading the 3-carbon SCFA propionate
were significantly depleted in children with higher cognitive function
scores across all age groups tested (Table 2; propionate degradation I,
under 6 months, enrichment score (E.S.) = -0.542, corrected p-value (q)
= 0.020; over 18 months, E.S. = -0.674, q = 0.041). Interestingly, genes
for propionate synthesis were also significantly depleted in higher
scoring children over 18 months (E.S. -0.676, q = 0.023), as were genes
for synthesizing the 2 carbon SCFA acetate in children under 6 months
old (acetate synthesis I, E.S. = -0.194, q = 0.153; acetate synthesis
II, E.S. = -0.342, q = 0.020; acetate synthesis III, E.S. = -0.31, q =
0.052). SCFAs are produced by anaerobic fermentation of dietary fiber
and have been linked with immune system regulation as well as directly
with brain function
\cite{dalileRoleShortchainFatty2019}.


Synthesis of menaquinone (vitamin K) was also negatively associated with
cognitive function score in older children (menaquinone synthesis I,
E.S. = -0.170, q = 0.0183). Menaquinone has several isoforms, one of
which, MK-4, has been found to be decreased in children diagnosed with ASD
\cite{dongCorrelationSerumConcentrations2021}
and is potentially neuroprotective in both rodents and humans
\cite{elkattawyVitaminK2Menaquinone72022}.
In children over 18 months old,
genes for the synthesis of the amino acids
glutamate and tryptophan were significantly enriched in children with
higher cognitive function scores (Glutamate synthesis I, E.S. = 0.242, q
= 0.047; Tryptophan synthesis, E.S. = 0.119, q = 0.041). Glutamate is a
critical neurotransmitter controlling neuronal excitatory/inhibitory
signaling along with gamma-aminobutyric acid (GABA), and their balance
in the brain controls neural plasticity and learning, particularly in
the developing brain
\cite{cohenkadoshLinkingGABAGlutamate2015,palomo-buitragoGlutamateInteractionsObesity2019}.
Tryptophan metabolism, including
microbial metabolism of tryptophan, has previously been linked with
autism in children
\cite{hoshinoBloodSerotoninFree1984,xiaoFecalMicrobiomeTransplantation2021}.
Taken together, these results suggest that
microbial metabolic activity, particularly the metabolism (synthesis and
degradation) of neuroactive compounds may have effects on cognitive
development.

\textbf{Table 2 - Feature set enrichment analysis on neuroactive
microbial gene sets}

\subsubsection{Gut microbial taxonomic and functional profiles predict cognitive function}

FSEA relies on understanding functional relationships between individual
genes. However, because the relationships between individual taxa are
still largely unknown, we turned to Random Forest (RF) models, an
unsupervised non-parametric machine learning (ML) method that enables
the identification of underlying patterns in large numbers of individual
features (here, microbial species). Previous studies have reported
successful use of RFs for processing highly-dimensional and sparse data
from the domain of genomics
\cite{amaratungaEnrichedRandomForests2008,brieucPracticalIntroductionRandom2018,chenRandomForestsGenomic2012,franzosaGutMicrobiomeStructure2019,stephanRandomForestApproach2015},
along with other works where it was used
to predict cognitive conditions related to Alzheimer's disease in
different scenarios
\cite{ardekaniPredictionIncipientAlzheimer2017,velazquezRandomForestModel2021}.
Additionally, given the sequential
nature of variable consideration in each tree, RFs are naturally able to
work out complex input feature interactions, such as those present in a
microbiome-wide study, without the necessity to explicitly compute
interaction terms.

\includegraphics[width=6.5in,height=5.19444in]{media/image2.png}

\textbf{Figure 3 - Random forest models predict concurrently measured
cognitive function.}

Comparison of RF predictor importance versus linear models for children
between birth and 6 months old (A), and for those older than 18 months
(B). Colors represent whether the species belong to the group of
top-important features that account for 80\% of the cumulative
importance on the RF model, if that species was significant (q
\textless{} 0.2) in linear models, both, or neither. Ranked feature
importance for taxa in RF models for children between birth and 6 months
old (C), and for those older than 18 months (D). Taxa that are important
for RF models both for children under 6 months and for those over 18
months (E), only for children under 6 months (F), and only for children
over 18 months (G)

Given that gut microbial profiles, as well as neurocognitive
development, may partially reflect socioeconomic and demographic
factors, we assessed the performance of RF regressors where maternal
education (a proxy of socioeconomic status (SES)), sex, and age were
included as possible predictors, either alone or in combination with
microbial taxonomic profiles (Table 3).

\textbf{Table 3. Benchmark metrics for the cognitive assessment score
prediction models. Confidence intervals are calculated from the
distribution of metrics from repeated CV at a confidence level of 95\%}


As with linear models, RF models for children under 6 months old were
less generalizable (mean test-set correlation -0.13, mean RMSE 12.70),
but RFs were consistently able to learn the relationship between taxa
and cognitive function scores in children over 18 months of age (mean
test-set correlation 0.429, mean RMSE 17.29). For both age groups,
species that were important in RF overlapped with those that were
significant when testing the relationship with linear models (Figure
3A-B, Supplementary Table XXA-B), though there were substantial
differences. For children under 6 months, only \emph{B. wexlerae} was
significantly associated with cognitive function in linear models, but
was ranked 20th in importance for RF models. The importance metric
employed (MDI) is a measure of how well the variable differentiates
(splits) sets of samples by creating subgroups that reduce the
intra-group deviations, while also accounting for the amount of samples
affected by a split. All taxa significantly associated with cognitive
function score in children over 18 months using LMs belong to the
top-ranking group responsible for 80\% of the total relative importance,
except for \emph{Eubacterium ramulus} (Table 3, Figure 3A).

Interestingly, several taxa highly ranked in importance in both age
groups, including several that were significant in LMs for children over
18 months old, including \emph{R. gnavus} (0--6 months, rank = 13;
18--120 months, rank = 7) and \emph{G. pamelaeae} (0--6 months, rank =
16; 18--120 months, rank = 20), while others such as \emph{Allistipes
finegoldii} were age-group specific. Several taxa important in RF models
were not statistically significant when using linear models after
multiple hypothesis correction. However, these taxa had small nominal
p-values. For example, \emph{Erysipelatoclostridium ramosum} was the
most important feature in RF models for children under 6 months old and
had an LM p-value of 0.04. Subject age was consistently ranked highly in
feature importance, which could indicate that decision branches based on
microbial taxa have increased purity when considering the subject's age
or that age itself is a useful predictor.


\subsubsection{Gut microbial taxonomic profiles predict brain structure differences}

If there are causal effects of microbial metabolism on cognitive
function, they might be reflected in changes in neuroanatomy. We again
employed a Random Forest modeling approach to associate gut taxonomic
profiles with individual brain regions identified in MRI scans,
normalized to total brain volume. Some brain regions were more readily
predicted by RF models trained on microbial taxa (Table 4, Supplementary
Table XX), in particular those that were highly correlated with age.
These included the L/R lingual gyrus (mean RF correlations, Left =
0.421, Right = 0.434; relative age importances, Left = 7.5\%, Right =
8.3\%) and the L/R pericalcarine cortex (mean RF correlations, Left =
0.200, Right = 0.273; relative age importances, Left = 3.7\%, Right =
6.3\%). In many cases, however, age was not an important variable in
high-performing models, such as that for the left accumbens area (mean
RF correlations = 0.288; relative age importances = 1.2\%).

\textbf{Table 4. Summary statistics for the neuroanatomy prediction
benchmarks}

We also observed that many brain regions had high accordance across the
left and right hemispheres in terms of both model performance and
microbial feature importance. In contrast, other regions had substantial
differences between the hemispheres. For example, the left accumbens
area, which plays an important role in reward circuits
\cite{ernstAmygdalaNucleusAccumbens2005,yauNucleusAccumbensResponse2012},
has one of the highest test-set correlations of our
brain region models (R = 0.288), as compared to the right accumbens
models which could not adequately generalize, and had a negative mean
test-set correlation (R = -0.041). Most healthy individuals have a
rightward asymmetry in the nucleus accumbens, and reduced asymmetry has
been linked to substance use disorder in young adults
\cite{caoMappingCorticalSubcortical2021}.

Feature importance for models of the left accumbens area were dominated
by three species of \emph{Bacteroides}, \emph{B. vulgatus} (3.4\%
relative importance), \emph{B. ovatus} (3.7\% relative importance), and
\emph{B. uniformis} (3.0\% relative importance).. The accumbens area is
associated with reward control, and in individuals diagnosed with ADHD,
it has been shown to have a divergent neuromorphology
\cite{hoogmanSubcorticalBrainVolume2017}.
Independently, \emph{B. ovatus}, \emph{B. uniformis} and
\emph{B. vulgatus} have been linked to ADHD
\cite{wangGutMicrobiotaDietary2020}.

In fact, there are studies showing that alterations on the striatal
dopamine transporters can cause effects resembling hyperactivity and
attention deficit
\cite{yaelDisinhibitionNucleusAccumbens2019},
and that \emph{B. uniformis} is gut-microbial modulator of the brain
dopamine transporter
\cite{hartstraInfusionDonorFeces2020}.


Many of the taxa identified included taxa also identified in LMs and RF
models of cognitive function. Interesting to note, while RF models for
multiple brain regions had many important microbial taxa, others were
dominated by a small number of taxa. In general, for all segments, a
consistent number of species between 14 and 22 (median = 20) was
responsible for one third of the fitness-weighted cumulative importance
(see Methods), regardless of the model benchmarks. We proceeded to
select a subset of the most important taxa (averaged over all segments)
and the segments whose importance was more heavily loaded to further
analyze the multiple relationships unveiled by the RF models
(Figure 4 A and B)

\includegraphics[width=6.5in,height=5.19444in]{media/image1.png}

\textbf{Figure 4: Microbial feature importance in predicting brain
volumes in children over 18 months of age.}

(A) Average test-set correlations for prediction of MRI segmentation
data from microbiome and demographics on select segments after repeated
cross validation. (B) Heatmap of average individual relative taxa
importances on each brain segment. Importances are reported as
proportions relative to the sum of importances for each model - since
every model is trained with 132 features, and even distribution of
importance would be 0.75\% for each feature. Segments and taxa ordered
by HCA on a select list of species with high ranks on average importances,
and their respective highest-load segments (C).

Our analysis revealed two major patterns of importance distribution from
taxa over the brain segments; some species portrayed high contributions
to multiple different segments, while others contributed modestly to
just one or two brain segments. Notable cases of the first pattern
included seven species - \emph{Anaerostipes hadrus}, \emph{Bacteroides
vulgatus}, \emph{Fusicatenibacter saccharivorans}, \emph{Ruminococcus
torques}, \emph{Eubacterium rectale}, \emph{Coprococcus comes} and
\emph{Blautia wexlerae} - that combined, account for approximately one
third of the cumulative relative importances, computed after subsetting
on the taxa of interest (Figure 4C).

Among these, the most important variable is \emph{Anaerostipes hadrus},
which is a butyrate-producing anaerobe that has been positively
associated with cognitive function
\cite{kantGenomeSequenceButyrateProducing2015,liCorrelationsGutMicrobiota2022}.
Its importance is, however, heavily loaded on the volume
of entire major lobes (frontal, parietal), which cannot be readily
correlated to specific cognitive functions. After accounting for the
major lobes, it is found to be highly important on the prediction of the
cerebellar vermal lobules VIII-X, pars opercularis, cuneus and
precuneus, and anterior-cingulate (Figure 4B).

To better understand this relationship, we performed hierarchical
cluster analysis, which revealed a well defined cluster of species with
high importance loadings on close segments of the lower temporal and
close occipital lobe, to which \emph{A. hadrus} belongs. The other
species on this cluster are \emph{B. wexlerae}, \emph{R. torques},
\emph{R. intestinalis, R. bicirculans} and \emph{F. saccahrivorans}, who
contribute heavily to the entorhinal, fusiform, lingual and
parahippocampal segments.

This group of taxa and their related segments drew our attention because
they contained \emph{B. wexlerae}, a highlight from the cognitive
assessment results. Previous works exploring Parkinson Disease patients
found out that issues in the cognitive task of confrontation naming were
positively correlated with thinning in the fusiform gyrus and
parahippocampal gyrus
\cite{pagonabarragaPatternRegionalCortical2013}.
Additionally, the left and right parahippocampal, where \emph{B.
wexlerae} had the highest importance in both models, are important in
visual/spatial processing and memory
\cite{aminoffRoleParahippocampalCortex2013}.
\emph{B. wexlerae} can also produce acetylcholine in the gut
\cite{hosomiOralAdministrationBlautia2022},
and this molecule plays an important role in modulating memory
function
\cite{haamCholinergicModulationHippocampal2017}.


Another notable importance cluster contains taxa associated with the
basal forebrain, especially the cingulate and the accumbens. This
cluster contains the previously-discussed \emph{B. ovatus} and \emph{B.
uniformis,} but also \emph{Alistipes finegoldii and Streptococcus
salivaris.} Reduction in the nucleus accumbens has been associated with
depression symptoms
\cite{wackerRoleNucleusAccumbens2009},
and increased levels of the \emph{Alistipes} genus have been
observed in patients with depressive disorder
\cite{jiangAlteredFecalMicrobiota2015}.

While these reports are important, simultaneously probing of the gut
metagenome and brain structure relationship is novel.

Both \emph{A. equolofaciens} and \emph{A. celatus,} two closely-related
equol-producing species, are examples of the second contribution
pattern, and had high importance in predicting the relative volume of
the right anterior cingulate (respectively, 3.0\% and 2.6\% relative
importances), which has been linked to social cognition and reward-based
decision making
\cite{appsAnteriorCingulateGyrus2016,boesRightAnteriorCingulate2008,bushDorsalAnteriorCingulate2002}.
Equol has a strong
estrogenic effect
\cite{setchellSEquolPotentLigand2005},
and in anterior cingulate, estrogen, has been shown to regulate
pain-related aversion
\cite{xiaoEstrogenAnteriorCingulate2013}.

Another example of this pattern is \emph{R. gnavus,} which was the only
species with significant negative association with cognitive function.
It is heavily associated with the left pars opercularis (2.7\% relative
importance), a relationship that may be explained by the emerging
understanding that this species is increased in individuals with insulin
resistance and obesity
\cite{leyHumanGutMicrobes2006},

conditions that are known to produce structural abnormalities in the
brain
\cite{opelBrainStructuralAbnormalities2021}.
Finally, \emph{Coprococcus comes} displays an importance
distribution that splits almost evenly among the two previously reported
clusters. Its loading on the prediction of the left posterior cingulate
is the highest for a microbe in RF models (4.0\% relative importance)
(only age had higher relative importance in any model), while also being
one of the most important predictors for neighboring areas of the
posterior cingulate such as the pars opercularis (relative importances,
Left = 2.3\%, Right = 2.8\%), along with upper regions like the left
precentral (2.3\% relative importance) and paracentral lobes (relative
importances, Left = 2.6\%, Right = 1.8\%).
