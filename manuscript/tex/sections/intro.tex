The gut and the brain are intimately linked. Signals from the brain
reach the gut through the autonomic nervous system and the endocrine
system, and the gut can communicate with the brain through the vagus
nerve and through endocrine and immune (cytokine) signaling molecules
\cite{cerdoEarlyNutritionGut2019,pronovostPerinatalInteractionsMicrobiome2019,sharonCentralNervousSystem2016,togniniGutMicrobiotaPotential2017}
In addition, the products of microbial metabolism generated in the gut can
influence the brain, both indirectly by stimulating the enteric nervous
and immune systems, as well as directly through molecules that enter
circulation and cross the blood-brain barrier. Causal links between the
gut microbiome and neural development, particularly atypical
development, are increasingly being identified
\cite{spichakMiningMicrobesMental2021}.
Both human epidemiology and animal models point to the effects
of gut microbes on the development of autism spectrum disorder
\cite{laueProspectiveAssociationsInfant2020,wanUnderdevelopmentGutMicrobiota2021}
and particular microbial taxa have been associated with depression
\cite{mayneris-perxachsMicrobiotaAlterationsProline2022,valles-colomerNeuroactivePotentialHuman2019}
and Alzheimer's disease
\cite{fungInteractionsMicrobiotaImmune2017,kimProbioticSupplementationImproves2021}.
But information about this "microbiome-gut-brain
axis" in normal neurocognitive development remains lacking,
particularly early in life.

The first years of life are critical developmental windows for both the
microbiome and the brain
\cite{laueDevelopingMicrobiomeBirth2022}.
Fetal development \emph{in utero} is believed to be mostly sterile, but
is rapidly seeded at birth through contact with the birth canal (if
birthed vaginally), caregivers, food sources (breastmilk or formula),
and other environmental sources such as antibiotics
\cite{backhedDynamicsStabilizationHuman2015,bokulichAntibioticsBirthMode2016}.
The early microbiome is characterized by low microbial
diversity, rapid succession and evolution, and domination by
Actinobacteria, particularly the genus \emph{Bifidobacterium},
Bacteroidetes, especially \emph{Bacteroides}, and Proteobacteria
\cite{koenigSuccessionMicrobialConsortia2011}.
Many of these microbes have specialized metabolic capabilities
for digesting human breast milk, such as \emph{Bifidobacterium infantis}
and \emph{Bacteroides fragilis}
\cite{selaGenomeSequenceBifidobacterium2008}.
Upon the introduction of solid foods, the gut
microbiome undergoes another categorical transformation, with most taxa
of the infant microbiome being replaced by taxa more reminiscent of
adult microbiomes
\cite{backhedDynamicsStabilizationHuman2015}.
Many prior studies have focused on either infant microbiomes or
adult microbiomes, since performing statistical analyses across this
transition poses particular challenges. Nevertheless, since this
transition coincides with critical neural developmental windows and
neural synaptogenesis, investigation across this solid-food boundary is
incredibly important
\cite{tauNormalDevelopmentBrain2010}.

A child's brain undergoes remarkable anatomical, microstructural,
organizational, and functional changes in the first years of life. By
age 5, a child's brain has reached \textgreater85\% of its adult size,
has achieved near-adult levels of myelination, and the pattern of axonal
connections has been established
\cite{silbereisCellularMolecularLandscapes2016}.
Much of this development occurs in discrete windows or
``sensitive-'' or ``critical periods'' (CPs)
\cite{knudsenSensitivePeriodsDevelopment2004}
when neural plasticity is particularly high,
and particular modes of learning
and skill development are preferred. The timing of these sensitive
periods is driven in part by genetics, but can also be affected by the
environment, including the gut microbiome
\cite{cowanAnnualResearchReview2020}.
In fact, emerging evidence suggests that the timing and duration of CPs
may be driven in part by cues from the developing gut microbiome
\cite{callaghanNestedSensitivePeriods2020}.
As such,
understanding the normal spectrum of healthy microbiome
development and how it relates to normal neurocognitive development may
provide opportunities for identifying atypical development earlier and
offer opportunities for intervention.

To begin to address this need, we investigated the gut microbiome and
neurocognitive development of children from infancy through 10 years of
age in an accelerated longitudinal study design. Gut microbial
communities were assessed using shotgun metagenomic sequencing, enabling
profiling at both the taxonomic and gene-functional level, and
neurocognitive development was measured using expert-assessment of
cognitive function and neuroimaging using magnetic resonance imaging
(MRI). Using a combination of classical statistical analysis and machine
learning, we showed that the development of the gut microbiome, the
brain, and children's cognitive abilities are intimately linked, with
both microbial taxa and gene functions able to predict cognitive
performance and brain structure.
