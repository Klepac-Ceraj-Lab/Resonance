\subsubsection{Study Ethics}

All procedures for this study were approved by the local institutional
review board at Rhode Island Hospital, and all experiments adhered to
the regulation of the review board. Written informed consent was
obtained from all parents or legal guardians of enrolled participants.

\subsubsection{Participants}

Data used in this study were drawn from the ongoing longitudinal
RESONANCE study of healthy and neurotypical brain and cognitive
development, based at Brown University in Providence, RI, USA. The
RESONANCE study is part of the NIH initiative Environmental influences
on Child Health Outcomes (ECHO)
\href{https://www.zotero.org/google-docs/?0tZanX}{(Forrest et al., 2018;
Gillman and Blaisdell, 2018)}, a longitudinal observational study of
healthy and neurotypical brain development that spans the fetal and
infant to adolescent life stages, combining neuroimaging (magnetic
resonance imaging, MRI), neurocognitive assessments, bio-specimen
analyses, subject genetics, environmental exposures such as lead, and
rich demographic, socioeconomic, family and medical history information.
From the RESONANCE cohort, 361 typically-developing children between the
ages of 2.8 months and 10 years old (median age 2 years, 2 months) were
selected for analysis in this study.

General participant demographics are provided in Table 1. Children are
representative of the RI population. Children in the RESONANCE cohort
were born full-term (\textless{} 37 weeks gestation) with height and
weight normal for gestational age, and from uncomplicated singleton
pregnancies. Children with known major risk factors for developmental
abnormalities at enrollment were excluded. In addition to screening at
the time of enrollment, on-going screening for worrisome behaviors using
validated tools was performed to identify at-risk children and remove
them from subsequent analysis.

Exclusion criteria included: \emph{in utero} exposure to alcohol,
cigarette or illicit substance exposure; preterm (\textless{} 37 weeks
gestation) birth; small for gestational age or less than 1500 g; fetal
ultrasound abnormalities; preeclampsia, high blood pressure, or
gestational diabetes; 5 minute APGAR scores \textless{} 8; NICU
admission; neurological disorder (e.g., head injury resulting in loss of
consciousness, epilepsy); and psychiatric or learning disorder
(including maternal depression) in the infant, parents, or siblings
requiring medication in the year prior to pregnancy.

Demographic and other non-biospecimen data such as race and ethnicity,
parental education and occupation, feeding behavior (breast- and
formula-feeding), child weight and height, were collected through
questionnaires or direct examination as appropriate. All data were
collected at every assessment visit, if possible.

\subsubsection{Cognitive Assessments}

Overall cognitive function was assessed using age-appropriate methods.
For children from birth to 30 months, we used an Early Learning
Composite as assessed via the Mullen Scales of Early Learning (MSEL)
\href{https://www.zotero.org/google-docs/?NSyykd}{(Mullen and others,
1995)}, a standardized and population-normed tool for assessing fine and
gross motor, expressive and receptive language, and visual reception
functioning in children from birth through 68 months of age. The
Wechsler Intelligence Quotient for Children (WISC)
\href{https://www.zotero.org/google-docs/?OopdA1}{(Wechsler, 2012)} is
an individually administered standard intelligence test for children
aged 6 to 16 years. It derives a full scale intelligence quotient (IQ)
score, which we used to assess overall cognitive functioning. The fourth
edition of the Wechsler Preschool and Primary Scale of Intelligence
(WPPSI-IV) is an individually administered standard intelligence test
for children aged 2 years 6 months to 7 years 7 months, trying to meet
the increasing need for the assessment of preschoolers. Just as the
WISC, it derives a full scale IQ score, which we used to assess overall
cognitive functioning.

\subsubsection{Stool Sample Collection an
Sequencing}

Stool samples (n=493) were collected by parents in OMR-200 tubes
(OMNIgene GUT, DNA Genotek, Ottawa, Ontario, Canada), stored on ice, and
brought within 24 hrs to the lab in RI where they were immediately
frozen at -80 ˚C. Stool samples were not collected if the subject had
taken antibiotics within the last two weeks. DNA extraction was
performed at Wellesley College (Wellesley, MA). Nucleic acids were
extracted from stool samples using the RNeasy PowerMicrobiome kit,
excluding the DNA degradation steps. Briefly, the samples were lysed by
bead beating using the Powerlyzer 24 Homogenizer (Qiagen, Germantown,
MD) at 2500 rpm for 45 s and then transferred to the QIAcube (Qiagen,
Germantown, MD) to complete the extraction protocol. Extracted DNA was
sequenced at the Integrated Microbiome Resource (IMR, Dalhousie
University, NS, Canada).

Shotgun metagenomic sequencing was performed on all samples. A pooled
library (max 96 samples per run) was prepared using the Illumina Nextera
Flex Kit for MiSeq and NextSeq from 1 ng of each sample. Samples were
then pooled onto a plate and sequenced on the Illumina NextSeq 550
platform using 150+150 bp paired-end ``high output'' chemistry,
generating 400 million raw reads and 120 Gb of sequence per plate.

\subsubsection{Machine learning for cognitiv
development}

Prediction of cognitive scores was carried out as a set of regression
experiments targeting real-valued continuous assessment scores.
Different experiment sets were designed to probe how different
representations of the gut microbiome (taxonomic profiles, Functional
profiles encoded as ECs) would behave, with and without the addition of
demographics (sex and maternal education as a proxy of socioeconomic
status) on participants from different age groups. Age (in months) was
provided as a covariate for all models (Table 3).

\textbf{Table 5. Experimental design and input composition for Random
Forest experiments}



Random Forests (RFs) \href{https://paperpile.com/c/dPbU4e/VDqU}{(Breiman
1996)} were selected as the prediction engine and processed using the
DecisionTree.jl
\href{https://www.zotero.org/google-docs/?dQLyDs}{(Sadeghi et al.,
2022)} implementation, inside the MLJ.jl
\href{https://www.zotero.org/google-docs/?y8ywGI}{(Blaom et al., 2020)}
Machine Learning framework. Independent RFs were trained for each
experiment, using a set of default regression hyperparameters from
Breiman and Cutler
\href{https://www.zotero.org/google-docs/?d3gqFh}{(Breiman, 2001)}, on a
repeated cross-validation approach with different RNG seeds. One hundred
repetitions of 3-fold CV with 10 different intra-fold RNG states each
were employed, for a total of 3000 experiments per input set.

After the training procedures, the root-mean-square error (RMSE) for
cognitive assessment scores and mean absolute proportional error (MAPE)
for the brain segmentation data, along with Pearson\textquotesingle s
correlation coefficient (R) were benchmarked on the validation and train
sets. MAPE was chosen as the metric for brain segments due to magnitude
differences between median volumes of each segment, which would hinder
interpretation of raw error values without additional reference.

To derive biological insight from the models, the covariate variable
importances for all the input features, measured by mean decrease in
impurity (MDI, or GINI importance), was also analyzed. Leveraging the
distribution of results from the extensive repeated cross validation
experiments, rather than electing a representative model or picking the
highest validation-set correlation, a measure of model fitness
(\textbf{Equation 1}) was designed to weight the importances from each
trained forest. The objective was to give more weight to those with
higher benchmarks on the validation sets (or higher generalizability),
while penalizing information from highly overfit models, drawing
inspiration from the approach used on another work employing repeated CV
on Random Forests with high-dimensional, low sample size microbiome
datasets \{***\}. The resulting fitness-weighted importances were used
to generate the values in Figure 3.

\(fitness =\)

\textbf{Equation 1.} Mathematical expression of the fitness measure used
to weight feature importances based on model benchmarks

\subsubsection{MRI / segmentation}

MRI data was acquired at a 3T Siemens Trio scanner with the following
parameters: TE=5.6msec, TR=1400msec, FA=15 degrees, 1.1x1.1x1.1mm
resolution, 160x160 matrix, with an average of 112 slices. FOV was
adjusted to infant size. Using a combination of linear and nonlinear
image registration, we created representative age-specific templates
using the ANTs package
\href{https://www.zotero.org/google-docs/?ZlWDfL}{(Avants et al.,
2014)}. After age specific templates were created, a single flow from
each age to the 12-month template was estimated and a final warp from
the 12-month template to standard adult MNI space was performed. For
each individual infant/ child brain image, we then calculated the warp
from their native T1w image space to their nearest-in-age template.
Using the resulting warps (native → nearest age template → 12-month
template → MNI template), we could move the standard adult brain atlas
to the space of an individual infant in a single step. In this case, we
used the Harvard-Oxford brain atlas to provide a coarse-grained
parcellation of individual brains into subcortical regions (e.g.,
thalamus, putamen) and total grey and white matter volumes (included as
part of the FSL package
\href{https://www.zotero.org/google-docs/?AapPN7}{(Jenkinson et al.,
2012)}. Total tissue and brainstem volume as well as left and right
hemisphere volumes were derived for total white and cortical gray
matter, lateral ventricle, thalamus, caudate, putamen, pallidum,
hippocampus, amygdala, and accumbens as well as total brainstem volume
\href{https://www.zotero.org/google-docs/?MCDDB9}{(Bruchhage et al.,
n.d.)}.

\subsubsection{Computational analysis of metagenomes}

Shotgun metagenomic sequences were analyzed using the bioBakery suite of
computational tools \href{https://paperpile.com/c/dPbU4e/vDau}{(Beghini
et al. 2021)}. First, KneadData (v0.7.7) was used to perform quality
control of raw sequence reads, such as read trimming and removal of
reads matching a human genome reference. Next, MetaPhlAn (v3.0.7, using
database mpa\_v30\_CHOCOPhlAn\_201901) was used to generate taxonomic
profiles by aligning reads to a reference database of marker genes.
Finally, HUMAnN (v3.0.0a4) was used to functionally profile the
metagenomes.

\subsubsection{Data and code availability}

Taxonomic and functional microbial profiles, as well as subject
demographics necessary for statistical analyses and machine learning are
available on the Open Science Framework
\href{https://www.zotero.org/google-docs/?61HLlM}{(Bonham et al.,
2022)}. Data processing, generation of summary statistics, and
generation of plots was performed using the julia programming language
\href{https://www.zotero.org/google-docs/?zj3HnY}{(Bezanson et al.,
2017; Bonham et al., 2021; Danisch and Krumbiegel, 2021)}. All code for
data analysis and figure generation, as well as scripts for automated
download of input files are available on github \{\{CITE zenodo\}\}.

\subsection{Research Standards}

\subsection{Acknowledgements}

Research funding: NIH UG3 OD023313 and Wellcome LEAP 1kD.