The relationship between the gut microbiome and brain function via the
gut-microbiome-brain axis has gained increasing acceptance largely as a
result of human epidemiological studies investigating atypical
neurocognition (eg anxiety and depression, neurodegeneration, attention
deficit / hyperactivity disorder, and autism) and mechanistic studies in
animal models. The results from these studies point to the possibility
that gut microbes and their metabolism may be causally implicated in
cognitive development, but this study is the first to our knowledge that
directly investigates microbial species and their genes in relation to
typical development in young children. Understanding the gut-brain-axis
in early life is particularly important, since differences or
interventions in early life can have outsized and longer-term
consequences than those at later ages. Further, even in the absence of
causal impacts of microbial metabolism, identifying risk factors that
could point to other early interventions would also have value.

The use of shotgun metagenomic sequencing enabled us to get
species-level resolution of microbial taxa. A previous study of
cognition in 3 year old subjects used 16S rRNA gene amplicon sequencing,
and showed that genera from the Lachnospiraceae family as well as
unclassified Clostridiales (now Eubacteriales) were associated with
higher scores on the Ages and Stages Questionnaire
\href{https://www.zotero.org/google-docs/?SB5Lwa}{(Sordillo et al.,
2019)}. However, each of these clades encompass dozens of genera with
diverse functions, each of which may have different effects. Indeed,
several of the taxa that were positively associated with cognitive
function in this study, including \emph{B. wexlerae}, the only species
identified by linear models in children under 6 months, \emph{D.
longicatena}, \emph{R. faecis}, and \emph{A. finegoldii} are
Clostridiales, as is \emph{R. gnavus}, which we found was negatively
associated with cognitive function (Figure 2A-B). This kind of
species-level resolution is typically not possible with amplicon
sequencing.

We identified several species in the family Eggerthelaceae that were
associated with cognitive function, including \emph{Gordonibacter
pamelaeae, Aldercreutzia equolofaciens, Asaccharobacter celatus}
(formerly regarded a subspecies of \emph{A. equolofaciens}
\href{https://www.zotero.org/google-docs/?OEhguI}{(Takahashi et al.,
2021)}), and \emph{Eggerthella lenta}. Many members of this family are
known in part due to unique metabolic activities. For example, \emph{A.
equolofaciens} produces the nonsteroidal estrogen equol from isoflaven
found in soybeans
\href{https://www.zotero.org/google-docs/?I2qNyB}{(Wang et al., 2005)},
and \emph{G. pamelaeae} can metabolize the polyphenol ellagic acid
(found in pomegranates and some berries) into urolithin, which has been
shown in some studies to have a neuroprotective effect
\href{https://www.zotero.org/google-docs/?oTN8Lw}{(Gong et al., 2022;
Selma et al., 2014)}. \emph{E. lenta} has been extensively studied for
its ability to metabolize drug compounds such as the plant-derived heart
medication digoxin
\href{https://www.zotero.org/google-docs/?tH2Qeq}{(Haiser et al.,
2013)}. The metabolic versatility of this clade, and the large number of
species that are associated with cognition make these microbes prime
targets for further mechanistic studies.

In addition to improved species-level resolution, shotgun metagenomic
sequencing also enables gene-functional insight. We showed here that
genes for the metabolism of SCFAs, both their degradation and synthesis,
are associated with cognitive function scores. However, while the
differential abundance of genes for the metabolism of neuroactive
compounds like these is suggestive, it is difficult to reason about the
relationship between levels of these genes and the gut concentrations of
the molecules their product enzymes act on. For example, while it might
be intuitive to reason that increased levels of menaquinone synthesis
genes is indicative of increased menaquinone, it could be the case that
menaquinone deficiency selects for microbes that can synthesize it. For
the same reason, increased propionate degradation genes may
counterintuitively be indicative of high levels of propionate in the gut
lumen, since high propionate would select for microbes that can
metabolize it. For this reason, future studies coupling shotgun
metagenomics with stool metabolomics could improve our understanding of
the relationship between microbial metabolism and cognitive development.
Further, strain-level analysis linking specific gene content in species
of interest could further refine targeted efforts at identifying
specific metabolic signatures of microbe-brain interactions.

The use of multiple age-appropriate cognitive assessments that could be
normalized to a common scale enabled us to analyze microbial
associations across multiple developmental periods, but carries several
drawbacks. In particular, the test-retest reliability, as well as
systematic differences between test administrators may introduce
substantial noise into these observations, particularly in the youngest
children. In addition, our study period overlapped with the beginning of
the COVID-19 pandemic, and we and others have observed some reduction in
measured scores for children that were assessed after the implementation
of lockdowns. In our subject set for this study, these effects are more
pronounced in some age groups due to our sampling schedule
\href{https://www.zotero.org/google-docs/?L8GnRj}{(Blackwell et al.,
2022; Deoni et al., 2021)} (Supplementary Figure 3).

This analysis allowed us to establish links between microbial taxa and
their functional potential with cognition and brain structure. Although
we cannot test causality or the chemistry behind the interactions
between gut microbial taxa, gut, and brain, this study provides clear
and statistically significant associations between the infant and early
child gut microbiota and neurocognition. Future studies should focus on
characterizing the early-life microbiome and neurocognitive development
across different geographic regions and lifestyles such as covering
traditionally understudied low-resource urban, peri-urban and rural
communities to obtain the more comprehensive understanding of the
variability within the different gut microbiomes reflects on
neurocognition. These studies would also provide us with the wealth of
data on different strains from the same species to better understand the
effect of genes and their products. Furthermore, culturing and microbial
community enrichment studies combined with genetic manipulation and
genomic approaches to understand microbial metabolism at the molecular
level is the key, as the metabolic functions shape and influence the
human host and its health. The discovery of the neuroactive metabolites
could provide us with biomarkers for early detection or necessary
medicinally useful molecules that can be applied in intervention.
