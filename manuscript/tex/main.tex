%%%%%%%%%%TO FORMAT A PREPRINT FOR A PCI%%%%%%%%%%%%%%%%
%%%%%%%%%%%%%%%%%%%%%%%%%%%%%%%%%%%%%%%%%%%%%%

\documentclass[a4paper]{article}

%%%%%%%%%%TO FORMAT A PREPRINT%%%%%%%%%%%%%%%%
%%%%%%%%%%%%%%%%%%%%%%%%%%%%%%%%%%%%%%%%%%%%%%
%%%%%%%%%%%%%%%%%%%%%%%%%%%%%%%%%%%%%%%%%%%%%%
  
\usepackage{lineno}
\usepackage{titlesec}
%\usepackage[none]{hyphenat} % use only if there is a problem
% Use Unicode characters
\usepackage[utf8]{inputenc}
% Clean citations with biblatex
\usepackage[
backend=biber,
natbib=true,
sortcites=true,
defernumbers=true,
style=authoryear,
citestyle=authoryear-comp,
maxnames=99,
maxcitenames=2,
giveninits=true,
terseinits=true,
url=false
]{biblatex}

% Highlighting
\usepackage{color}
\usepackage{soul}
\usepackage{xurl}
\usepackage{longtable,booktabs,array}
\usepackage{multirow}
\usepackage{graphicx}

% Headers and footers

\title{Gut microbes and their genes or something else}
%%%%%%%%%%%%%%%%%%%%%%%%%%%%%%%%%%%%%%%%%%%%%%%%%%%%%%%%%%%%%%%%%%%%%%%%%%%%%

%%%%%%%%%  SET THE LIST OF AUTHORS WITH CORRESPONDING AFFILIATIONS  use \& before last author %%%%%%%%%%%%%%%%%%%%
\author{Kevin S. Bonham\textsuperscript{1}, \space
Guilherme Fahur Bottino\textsuperscript{1}, 
Firstname Lastname\textsuperscript{2}, \&
Vanja Klepac-Ceraj\textsuperscript{1}
%\fourthauthor \textsuperscript{i}
%etc...
}
%%%%%%%%%%%%%%%%%%%%%%%%%%%%%%%%%%%%%%%%%%%%%%%%%%%%%%%%%%%%%%%%%%%%%%%%%%%%%

%%%%%%%%%%%%%%%%%%%%%%%%%%%%%%%%%  SET THE ABSTRACT %%%%%%%%%%%%%%%%%

\begin{document}

\begin{abstract}
    The gastrointestinal tract, its resident microorganisms and the central
    nervous system are connected by biochemical signaling, also known as
    "microbiome-gut-brain-axis." Both the human brain and the gut microbiome
    have critical developmental windows in the first three years of life,
    raising the possibility that their development is co-occurring and
    likely co-dependent. Emerging evidence implicates gut microorganisms and
    microbiota composition in cognitive outcomes and neurodevelopmental
    disorders (e.g., autism and anxiety), but the influence of gut microbial
    metabolism on typical neurodevelopment has not been explored in detail.
    We investigated the relationship of the microbiome with the neuroanatomy
    and cognitive function of 361 healthy children, demonstrating that
    differences in gut microbial taxa and gene functions are associated with
    overall cognitive function as well as with differences in the size of
    multiple brain regions. Using a combination of multivariate linear
    models and machine learning (ML) models, we showed that many species,
    including \emph{Gordonibacter pamelae} and \emph{Blautia wexlerae}, were
    significantly associated with higher cognitive function, while some
    species such as \emph{Ruminococcus gnavus} were more commonly found in
    children with low cognitive scores regardless of age or maternal
    education. Microbial genes for enzymes involved in the metabolism of
    neuroactive compounds, particularly short-chain fatty acids such as
    glutamate and propionate, were also associated with cognitive function.
    In addition, ML models were able to use microbial taxa to predict the
    volume of brain regions, with particular taxa often dominating the
    feature importance metric for individual brain regions, such as \emph{B.
    wexlerae} being the most dominant for the parahippocampal region or
    GABA-producing \emph{Bacteroidetes ovatus} for left accumbens. These
    findings provide potential biomarkers of neurocognition and may lead to
    the future development of targets for early detection and early
    intervention.
\end{abstract}

%%%%%%%%%%%%%%%%%%%%%%%%%%%% Text of the preprint %%%%%%%%%%%%%%%%%%%%%%%%%%%%%%
%%%%%%%%%%%%%%%%%%%%%%%%%%%%%%%%%%%%%%%%%%%%%%%%%%%%%%%%%%%%%%%%%%%%%%%%%%%%%%%%
% use \emph{} for italics and \parencite{} to cite a reference and \ref{XX} to cite the \label{XX}
% use \\ and a blank line to mark the end of paragraph and starting of a new one

\section*{Introduction}

The gut and the brain are intimately linked. Signals from the brain
reach the gut through the autonomic nervous system and the endocrine
system, and the gut can communicate with the brain through the vagus
nerve and through endocrine and immune (cytokine) signaling molecules
\cite{cerdoEarlyNutritionGut2019,pronovostPerinatalInteractionsMicrobiome2019,sharonCentralNervousSystem2016,togniniGutMicrobiotaPotential2017}
In addition, the products of microbial metabolism generated in the gut can
influence the brain, both indirectly by stimulating the enteric nervous
and immune systems, as well as directly through molecules that enter
circulation and cross the blood-brain barrier. Causal links between the
gut microbiome and neural development, particularly atypical
development, are increasingly being identified
\cite{spichakMiningMicrobesMental2021}.
Both human epidemiology and animal models point to the effects
of gut microbes on the development of autism spectrum disorder
\cite{laueProspectiveAssociationsInfant2020,wanUnderdevelopmentGutMicrobiota2021}
and particular microbial taxa have been associated with depression
\cite{mayneris-perxachsMicrobiotaAlterationsProline2022,valles-colomerNeuroactivePotentialHuman2019}
and Alzheimer's disease
\cite{fungInteractionsMicrobiotaImmune2017,kimProbioticSupplementationImproves2021}.
But information about this "microbiome-gut-brain
axis" in normal neurocognitive development remains lacking,
particularly early in life.

The first years of life are critical developmental windows for both the
microbiome and the brain
\cite{laueDevelopingMicrobiomeBirth2022}.
Fetal development \emph{in utero} is believed to be mostly sterile, but
is rapidly seeded at birth through contact with the birth canal (if
birthed vaginally), caregivers, food sources (breastmilk or formula),
and other environmental sources such as antibiotics
\cite{backhedDynamicsStabilizationHuman2015,bokulichAntibioticsBirthMode2016}.
The early microbiome is characterized by low microbial
diversity, rapid succession and evolution, and domination by
Actinobacteria, particularly the genus \emph{Bifidobacterium},
Bacteroidetes, especially \emph{Bacteroides}, and Proteobacteria
\cite{koenigSuccessionMicrobialConsortia2011}.
Many of these microbes have specialized metabolic capabilities
for digesting human breast milk, such as \emph{Bifidobacterium infantis}
and \emph{Bacteroides fragilis}
\cite{selaGenomeSequenceBifidobacterium2008}.
Upon the introduction of solid foods, the gut
microbiome undergoes another categorical transformation, with most taxa
of the infant microbiome being replaced by taxa more reminiscent of
adult microbiomes
\cite{backhedDynamicsStabilizationHuman2015}.
Many prior studies have focused on either infant microbiomes or
adult microbiomes, since performing statistical analyses across this
transition poses particular challenges. Nevertheless, since this
transition coincides with critical neural developmental windows and
neural synaptogenesis, investigation across this solid-food boundary is
incredibly important
\cite{tauNormalDevelopmentBrain2010}.

A child's brain undergoes remarkable anatomical, microstructural,
organizational, and functional changes in the first years of life. By
age 5, a child's brain has reached \textgreater85\% of its adult size,
has achieved near-adult levels of myelination, and the pattern of axonal
connections has been established
\cite{silbereisCellularMolecularLandscapes2016}.
Much of this development occurs in discrete windows or
``sensitive-'' or ``critical periods'' (CPs)
\cite{knudsenSensitivePeriodsDevelopment2004}
when neural plasticity is particularly high,
and particular modes of learning
and skill development are preferred. The timing of these sensitive
periods is driven in part by genetics, but can also be affected by the
environment, including the gut microbiome
\cite{cowanAnnualResearchReview2020}.
In fact, emerging evidence suggests that the timing and duration of CPs
may be driven in part by cues from the developing gut microbiome
\cite{callaghanNestedSensitivePeriods2020}.
As such,
understanding the normal spectrum of healthy microbiome
development and how it relates to normal neurocognitive development may
provide opportunities for identifying atypical development earlier and
offer opportunities for intervention.

To begin to address this need, we investigated the gut microbiome and
neurocognitive development of children from infancy through 10 years of
age in an accelerated longitudinal study design. Gut microbial
communities were assessed using shotgun metagenomic sequencing, enabling
profiling at both the taxonomic and gene-functional level, and
neurocognitive development was measured using expert-assessment of
cognitive function and neuroimaging using magnetic resonance imaging
(MRI). Using a combination of classical statistical analysis and machine
learning, we showed that the development of the gut microbiome, the
brain, and children's cognitive abilities are intimately linked, with
both microbial taxa and gene functions able to predict cognitive
performance and brain structure.


\section*{Results}

\subsection*{Outline}

\begin{itemize}
  \item Intial description: Cohort size, demographics, data types
  \item Basic analysis: Summary statistics, ordinations, permanovas, mantel tests
  \item Exploration of taxonomic profiles + cogscores / brain, cross-sectional. Linear models. \hl{Is this necessary??}
  \item Try HAllA on multivariate data?
  \item Predictive model (RF) - taxonomic profiles + cogscores.
  \item Taxonomic profiles + brain structure
  \item Gene functions + metabolomes
\end{itemize}

\section*{Discussion}

The relationship between the gut microbiome and brain function via the
gut-microbiome-brain axis has gained increasing acceptance largely as a
result of human epidemiological studies investigating atypical
neurocognition (eg anxiety and depression, neurodegeneration, attention
deficit / hyperactivity disorder, and autism) and mechanistic studies in
animal models. The results from these studies point to the possibility
that gut microbes and their metabolism may be causally implicated in
cognitive development, but this study is the first to our knowledge that
directly investigates microbial species and their genes in relation to
typical development in young children. Understanding the gut-brain-axis
in early life is particularly important, since differences or
interventions in early life can have outsized and longer-term
consequences than those at later ages. Further, even in the absence of
causal impacts of microbial metabolism, identifying risk factors that
could point to other early interventions would also have value.

The use of shotgun metagenomic sequencing enabled us to get
species-level resolution of microbial taxa. A previous study of
cognition in 3 year old subjects used 16S rRNA gene amplicon sequencing,
and showed that genera from the Lachnospiraceae family as well as
unclassified Clostridiales (now Eubacteriales) were associated with
higher scores on the Ages and Stages Questionnaire
\cite{sordilloAssociationInfantGut2019}.
However, each of these clades encompass dozens of genera with
diverse functions, each of which may have different effects. Indeed,
several of the taxa that were positively associated with cognitive
function in this study, including \emph{B. wexlerae}, the only species
identified by linear models in children under 6 months, \emph{D.
longicatena}, \emph{R. faecis}, and \emph{A. finegoldii} are
Clostridiales, as is \emph{R. gnavus}, which we found was negatively
associated with cognitive function (Figure 2A-B). This kind of
species-level resolution is typically not possible with amplicon
sequencing.

We identified several species in the family Eggerthelaceae that were
associated with cognitive function, including \emph{Gordonibacter
pamelaeae, Aldercreutzia equolofaciens, Asaccharobacter celatus}
(formerly regarded a subspecies of \emph{A. equolofaciens}
\cite{takahashiCompleteGenomeSequence2021}),
and \emph{Eggerthella lenta}. Many members of this family are
known in part due to unique metabolic activities. For example, \emph{A.
equolofaciens} produces the nonsteroidal estrogen equol from isoflaven
found in soybeans
\cite{wangEnantioselectiveSynthesisSEquol2005},
and \emph{G. pamelaeae} can metabolize the polyphenol ellagic acid
(found in pomegranates and some berries) into urolithin, which has been
shown in some studies to have a neuroprotective effect
\cite{gongUrolithinAlleviatesBloodbrain2022,selmaDescriptionUrolithinProduction2014}.
\emph{E. lenta} has been extensively studied for
its ability to metabolize drug compounds such as the plant-derived heart
medication digoxin
\cite{haiserPredictingManipulatingCardiac2013}.
The metabolic versatility of this clade, and the large number of
species that are associated with cognition make these microbes prime
targets for further mechanistic studies.

In addition to improved species-level resolution, shotgun metagenomic
sequencing also enables gene-functional insight. We showed here that
genes for the metabolism of SCFAs, both their degradation and synthesis,
are associated with cognitive function scores. However, while the
differential abundance of genes for the metabolism of neuroactive
compounds like these is suggestive, it is difficult to reason about the
relationship between levels of these genes and the gut concentrations of
the molecules their product enzymes act on. For example, while it might
be intuitive to reason that increased levels of menaquinone synthesis
genes is indicative of increased menaquinone, it could be the case that
menaquinone deficiency selects for microbes that can synthesize it. For
the same reason, increased propionate degradation genes may
counterintuitively be indicative of high levels of propionate in the gut
lumen, since high propionate would select for microbes that can
metabolize it. For this reason, future studies coupling shotgun
metagenomics with stool metabolomics could improve our understanding of
the relationship between microbial metabolism and cognitive development.
Further, strain-level analysis linking specific gene content in species
of interest could further refine targeted efforts at identifying
specific metabolic signatures of microbe-brain interactions.

The use of multiple age-appropriate cognitive assessments that could be
normalized to a common scale enabled us to analyze microbial
associations across multiple developmental periods, but carries several
drawbacks. In particular, the test-retest reliability, as well as
systematic differences between test administrators may introduce
substantial noise into these observations, particularly in the youngest
children. In addition, our study period overlapped with the beginning of
the COVID-19 pandemic, and we and others have observed some reduction in
measured scores for children that were assessed after the implementation
of lockdowns. In our subject set for this study, these effects are more
pronounced in some age groups due to our sampling schedule
\cite{blackwellYouthWellbeingCOVID192022,deoniImpactCOVID19Pandemic2021}
(Supplementary Figure 3).

This analysis allowed us to establish links between microbial taxa and
their functional potential with cognition and brain structure. Although
we cannot test causality or the chemistry behind the interactions
between gut microbial taxa, gut, and brain, this study provides clear
and statistically significant associations between the infant and early
child gut microbiota and neurocognition. Future studies should focus on
characterizing the early-life microbiome and neurocognitive development
across different geographic regions and lifestyles such as covering
traditionally understudied low-resource urban, peri-urban and rural
communities to obtain the more comprehensive understanding of the
variability within the different gut microbiomes reflects on
neurocognition. These studies would also provide us with the wealth of
data on different strains from the same species to better understand the
effect of genes and their products. Furthermore, culturing and microbial
community enrichment studies combined with genetic manipulation and
genomic approaches to understand microbial metabolism at the molecular
level is the key, as the metabolic functions shape and influence the
human host and its health. The discovery of the neuroactive metabolites
could provide us with biomarkers for early detection or necessary
medicinally useful molecules that can be applied in intervention.


\section*{Material and methods}

\subsection*{Human Subjects}

Data used in this study were drawn from the ongoing longitudinal RESONANCE study
of healthy and neurotypical brain and cognitive development,
based at Brown University in Providence, RI, USA.
RESONANCE is part of the NIH initiative Environmental influences on Child Health Outcomes (ECHO) \cite{Forrest2018-ud,Gillman2018-om},
a longitudinal observational study of healthy and neurotypical brain development
that spans the fetal and infant to adolescent life stages,
combining neuroimaging (magnetic resonance imaging, MRI), neurocognitive assessments, bio-specimen analyses, subject genetics,
environmental exposures such as lead, and rich demographic, socioeconomic, family and medical history information.
From the RESONANCE cohort, 344 typically-developing children
between the ages of 28 days and 15 years old were selected for analysis in this study. 

General participant demographics are provided in Tables \ref{tab:demographics} and \ref{tab:agestats} and Figure \ref{fig:data}.
Complete metadata are available in Data Records (see below), with children being representative of the RI population.
As a broad background, children in the RESONANCE cohort were born full-term (>37 weeks gestation)
with height and weight normal for gestational age, and from uncomplicated singleton pregnancies.
Children with known major risk factors for developmental abnormalities at enrollment were excluded.
In addition to screening at the time of enrollment,
on-going screening for worrisome behaviors using validated tools was performed
to identify at-risk children and remove them from subsequent analysis.

Exclusion criteria included: \emph{in utero} exposure to alcohol, cigarette or illicit substance exposure;
preterm (<37 wks gestation) birth; small for gestational age or less than 1500 g; fetal ultrasound abnormalities;
preeclampsia, high blood pressure, or gestational diabetes; 5 minute APGAR scores <8;
NICU admission; neurological disorder (e.g., head injury resulting in loss of consciousness, epilepsy);
and psychiatric or learning disorder (including maternal depression) in the infant, parents, or siblings requiring medication in the year prior to pregnancy.

Demographic and other non-biospecimen data such as race and ethnicity, parental education and occupation,
feeding behavior (breast- and formula-feeding), child weight and height,
were collected through questionnaires or direct examination as appropriate.
All data were collected at every assessment visit.
All procedures for this study were approved by the local institutional review board at Rhode Island Hospital,
and all experiments adhered to the regulation of the review board.
Written informed consent was obtained from all parents or legal guardians of enrolled participants.


\subsection*{Cognitive Assessments}

Overall cognitive function was assessed using age-appropriate methods.
For children from birth to 30 months, we used an Early Learning Composite
as assessed via the Mullen Scales of Early Learning (MSEL) \cite{Mullen1995-ty},
a standardized and population-normed tool for assessing fine and gross motor,
expressive and receptive language, and visual reception functioning in children from birth through 68 months of age.

The third edition of the Bayley Scales of Infant and Toddler Development \cite{Bayley2006-wm}
is a standard series of measures used primarily to assess the development of infants and toddlers,
ranging from 1 to 42 months of age.

The Wechsler Intelligence Quotient for Children (WISC) \cite{Wechsler2012-mi}
is an individually administered standard intelligence test for children aged 6 to 16 years.
It derives a full scale intelligence quotient (IQ) score, which we used to assess overall cognitive functioning.
The fourth edition of the Wechsler Preschool and Primary Scale of Intelligence (WPPSI-IV) \cite{Wechsler2012-mi}
is an individually administered standard intelligence test for children aged 2 years 6 months to 7 years 7 months,
trying to meet the increasing need for the assessment of preschoolers.
Just as the WISC, it derives a full scale IQ score, which we used to assess overall cognitive functioning.


\subsection*{Stool Sample Collection and Sequencing}

Stool samples (n=493) were collected by parents in OMR-200 tubes (OMNIgene GUT, DNA Genotek, Ottawa, Ontario, Canada),
immediately stored on ice, and brought within 24 hrs to the lab in RI where they were immediately frozen at -80 $^{\circ}$C.
Stool samples were not collected if the subject had taken antibiotics within the last two weeks.
DNA extraction was performed at Wellesley College (Wellesley, MA).
Nucleic acids were extracted from stool samples using the RNeasy PowerMicrobiome kit
automated on the QIAcube (Qiagen, Germantown, MD), excluding the DNA degradation steps.
Extracted DNA was sequenced at the Integrated Microbiome Resource (IMR, Dalhousie University, NS, Canada)

Shotgun metagenomic sequencing was performed on all samples.
A pooled library (max 96 samples per run) was prepared using the Illumina Nextera Flex Kit for MiSeq and NextSeq from 1 ng of each sample.
Samples were then pooled onto a plate and sequenced
on the Illumina NextSeq 550 platform using 150+150 bp paired-end “high output” chemistry,
generating ~400 million raw reads and ~120 Gb of sequence per plate.

\hl{Are we still planning to include 16S data here?}

For sequencing 16S rRNA gene amplicons,
the V4-V5 region of the 16S ribosomal RNA gene was sequenced according to the protocol
described by Comeau et al. \cite{Comeau2017-jg}.
Briefly, the V4-V5 region was amplified once using the Phusion High-Fidelity DNA polymerase
(ThermoFisher Scientific, Waltham, MA) and universal bacterial primers
515F: 5’-GTGYCAGCMGCCGCGGTAA-3’ and 926R: 5’-CCGYCAATTYMTTTRAGTTT-3’ \cite{Parada2016-uz,Walters2016-fi}.
These primers had appropriate Illumina adapters and error-correcting barcodes unique to each sample
to allow up to 380 samples to be simultaneously run per single flow cell.
After being pooled into a single library and quantified fluorometrically,
samples were cleaned-up and normalized using the high-throughput Charm Biotech Just-a-Plate 96-well Normalization Kit (Charm Biotech, Cape Girardeau, MO).
The normalized samples were sequenced on the Illumina MiSeq platform (Illumina, San Diego, CA)
using 300+300 bp paired-end V3 chemistry, producing ~55,000 raw reads per sample.

\subsection*{Computational Analysis}

Shotgun metagenomic sequences were analyzed using the bioBakery suite of computational tools \cite{McIver2018-yc}.
First, \verb|KneadData| (v0.7.7) was used to perform quality control of raw sequence reads,
such as read trimming and removal of reads matching a human genome reference.
Next, \verb|MetaPhlAn| (v3.0.7, using database \verb|mpa_v30_CHOCOPhlAn_201901|) was used to generate taxonomic profiles
by aligning reads to a reference database of marker genes.
Finally, \verb|HUMAnN| (v3.0.0a4) was used to functionally profile the metagenomes.

Raw amplicon sequences were profiled using Quantitative Insights in Microbial Ecology 2 (QIIME2) v2021.2.0 \cite{Bolyen2019-qq}.
Briefly, primers flanking V4-V5 were removed from fastq reads using the cutadapt (v3.2) QIIME2 plugin \cite{Martin2011-zv}.
Raw sequence reads of samples from enrichment cultures were denoised, filtered and clustered into amplicon sequence variants (ASVs) using the Divisive Amplicon Denoising Algorithm (DADA2) plugin in QIIME 2 \cite{Callahan2016-ol}.
After denoising and filtering, 16270 total sequences were recovered with a mean length of 373 bases (270-465, standard deviation 13.21).
Taxonomy was assigned to each ASV using a Naïve-Bayes classifier compared against SILVA v.138 reference database \cite{Yilmaz2013-rj,Quast2013-hc}
trained on the 515F-806R region of the 16S rRNA gene \cite{Bokulich2018-dv}.

Additional data processing, generation of summary statistics, 
and generation of plots was performed using the julia programming language \cite{Bezanson2017-ud}.
See Code Availability section for additional details.



\end{document}
