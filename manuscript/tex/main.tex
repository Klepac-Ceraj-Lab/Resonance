%%%%%%%%%%TO FORMAT A PREPRINT FOR A PCI%%%%%%%%%%%%%%%%
%%%%%%%%%%%%%%%%%%%%%%%%%%%%%%%%%%%%%%%%%%%%%%

\documentclass[a4paper]{article}

\usepackage{lineno}
\usepackage{titlesec}
%\usepackage[none]{hyphenat} % use only if there is a problem
% Use Unicode characters
\usepackage[utf8]{inputenc}
% Clean citations with biblatex
\usepackage[
backend=biber,
natbib=true,
sortcites=true,
defernumbers=true,
style=authoryear,
citestyle=authoryear-comp,
maxnames=99,
maxcitenames=2,
giveninits=true,
terseinits=true,
url=false
]{biblatex}

\addbibresource{zotero_library}

% Highlighting
\usepackage{color}
\usepackage{soul}
\usepackage{xurl}
\usepackage{longtable,booktabs,array}
\usepackage{multirow}
\usepackage{graphicx}

\title{Gut microbes and their genes or something else}
%%%%%%%%%%%%%%%%%%%%%%%%%%%%%%%%%%%%%%%%%%%%%%%%%%%%%%%%%%%%%%%%%%%%%%%%%%%%%

%%%%%%%%%  SET THE LIST OF AUTHORS WITH CORRESPONDING AFFILIATIONS  use \& before last author %%%%%%%%%%%%%%%%%%%%
\author{Kevin S. Bonham\textsuperscript{1}, \space
Guilherme Fahur Bottino\textsuperscript{1}, 
Firstname Lastname\textsuperscript{2}, \&
Vanja Klepac-Ceraj\textsuperscript{1}
%\fourthauthor \textsuperscript{i}
%etc...
}
%%%%%%%%%%%%%%%%%%%%%%%%%%%%%%%%%%%%%%%%%%%%%%%%%%%%%%%%%%%%%%%%%%%%%%%%%%%%%

%%%%%%%%%%%%%%%%%%%%%%%%%%%%%%%%%  SET THE ABSTRACT %%%%%%%%%%%%%%%%%

\begin{document}

\begin{abstract}
The gastrointestinal tract, its resident microorganisms and the central
nervous system are connected by biochemical signaling, also known as
"microbiome-gut-brain-axis." Both the human brain and the gut microbiome
have critical developmental windows in the first three years of life,
raising the possibility that their development is co-occurring and
likely co-dependent. Emerging evidence implicates gut microorganisms and
microbiota composition in cognitive outcomes and neurodevelopmental
disorders (e.g., autism and anxiety), but the influence of gut microbial
metabolism on typical neurodevelopment has not been explored in detail.
We investigated the relationship of the microbiome with the neuroanatomy
and cognitive function of 361 healthy children, demonstrating that
differences in gut microbial taxa and gene functions are associated with
overall cognitive function as well as with differences in the size of
multiple brain regions. Using a combination of multivariate linear
models and machine learning (ML) models, we showed that many species,
including \emph{Gordonibacter pamelae} and \emph{Blautia wexlerae}, were
significantly associated with higher cognitive function, while some
species such as \emph{Ruminococcus gnavus} were more commonly found in
children with low cognitive scores regardless of age or maternal
education. Microbial genes for enzymes involved in the metabolism of
neuroactive compounds, particularly short-chain fatty acids such as
glutamate and propionate, were also associated with cognitive function.
In addition, ML models were able to use microbial taxa to predict the
volume of brain regions, with particular taxa often dominating the
feature importance metric for individual brain regions, such as \emph{B.
wexlerae} being the most dominant for the parahippocampal region or
GABA-producing \emph{Bacteroidetes ovatus} for left accumbens. These
findings provide potential biomarkers of neurocognition and may lead to
the future development of targets for early detection and early
intervention.
\end{abstract}

%%%%%%%%%%%%%%%%%%%%%%%%%%%% Text of the preprint %%%%%%%%%%%%%%%%%%%%%%%%%%%%%%
%%%%%%%%%%%%%%%%%%%%%%%%%%%%%%%%%%%%%%%%%%%%%%%%%%%%%%%%%%%%%%%%%%%%%%%%%%%%%%%%
% use \emph{} for italics and \parencite{} to cite a reference and \ref{XX} to cite the \label{XX}
% use \\ and a blank line to mark the end of paragraph and starting of a new one

\section*{Introduction}

The gut and the brain are intimately linked. Signals from the brain
reach the gut through the autonomic nervous system and the endocrine
system, and the gut can communicate with the brain through the vagus
nerve and through endocrine and immune (cytokine) signaling molecules
\cite{cerdoEarlyNutritionGut2019,pronovostPerinatalInteractionsMicrobiome2019,sharonCentralNervousSystem2016,togniniGutMicrobiotaPotential2017}
In addition, the products of microbial metabolism generated in the gut can
influence the brain, both indirectly by stimulating the enteric nervous
and immune systems, as well as directly through molecules that enter
circulation and cross the blood-brain barrier. Causal links between the
gut microbiome and neural development, particularly atypical
development, are increasingly being identified \cite{spichakMiningMicrobesMental2021}.
Both human epidemiology and animal models point to the effects
of gut microbes on the development of autism spectrum disorder
\cite{laueProspectiveAssociationsInfant2020,wanUnderdevelopmentGutMicrobiota2021}
and particular microbial taxa have been associated with depression
\cite{mayneris-perxachsMicrobiotaAlterationsProline2022,valles-colomerNeuroactivePotentialHuman2019}
and Alzheimer's disease \cite{fungInteractionsMicrobiotaImmune2017,kimProbioticSupplementationImproves2021}.
But information about this "microbiome-gut-brain
axis" in normal neurocognitive development remains lacking,
particularly early in life.

The first years of life are critical developmental windows for both the
microbiome and the brain
\href{https://www.zotero.org/google-docs/?XTqhlm}{(Laue et al., 2022)}.
Fetal development \emph{in utero} is believed to be mostly sterile, but
is rapidly seeded at birth through contact with the birth canal (if
birthed vaginally), caregivers, food sources (breastmilk or formula),
and other environmental sources such as antibiotics
\href{https://www.zotero.org/google-docs/?X6N0f4}{(Bäckhed et al., 2015;
Bokulich et al., 2016; Dominguez-Bello et al., 2010; Louwies et al.,
2020)}. The early microbiome is characterized by low microbial
diversity, rapid succession and evolution, and domination by
Actinobacteria, particularly the genus \emph{Bifidobacterium},
Bacteroidetes, especially \emph{Bacteroides}, and Proteobacteria
\href{https://www.zotero.org/google-docs/?VG43uh}{(Koenig et al.,
2011)}. Many of these microbes have specialized metabolic capabilities
for digesting human breast milk, such as \emph{Bifidobacterium infantis}
and \emph{Bacteroides fragilis}
\href{https://www.zotero.org/google-docs/?1dKzKj}{(Sela et al., 2008;
Tso et al., 2021)}. Upon the introduction of solid foods, the gut
microbiome undergoes another categorical transformation, with most taxa
of the infant microbiome being replaced by taxa more reminiscent of
adult microbiomes
\href{https://www.zotero.org/google-docs/?V3lY3p}{(Bäckhed et al.,
2015)}. Many prior studies have focused on either infant microbiomes or
adult microbiomes, since performing statistical analyses across this
transition poses particular challenges. Nevertheless, since this
transition coincides with critical neural developmental windows and
neural synaptogenesis, investigation across this solid-food boundary is
incredibly important
\href{https://www.zotero.org/google-docs/?d29bv8}{(Tau and Peterson,
2010)}.

A child's brain undergoes remarkable anatomical, microstructural,
organizational, and functional changes in the first years of life. By
age 5, a child's brain has reached \textgreater85\% of its adult size,
has achieved near-adult levels of myelination, and the pattern of axonal
connections has been established
\href{https://www.zotero.org/google-docs/?1jUZOO}{(Silbereis et al.,
2016)}. Much of this development occurs in discrete windows or
``sensitive-'' or ``critical periods'' (CPs)
\href{https://www.zotero.org/google-docs/?w4Pcze}{(Knudsen, 2004)} when
neural plasticity is particularly high, and particular modes of learning
and skill development are preferred. The timing of these sensitive
periods is driven in part by genetics, but can also be affected by the
environment, including the gut microbiome
\href{https://www.zotero.org/google-docs/?KTm9lQ}{(Cowan et al., 2020)}.
In fact, emerging evidence suggests that the timing and duration of CPs
may be driven in part by cues from the developing gut microbiome
\href{https://www.zotero.org/google-docs/?mAD1ne}{(Callaghan, 2020)}. As
such, understanding the normal spectrum of healthy microbiome
development and how it relates to normal neurocognitive development may
provide opportunities for identifying atypical development earlier and
offer opportunities for intervention.

To begin to address this need, we investigated the gut microbiome and
neurocognitive development of children from infancy through 10 years of
age in an accelerated longitudinal study design. Gut microbial
communities were assessed using shotgun metagenomic sequencing, enabling
profiling at both the taxonomic and gene-functional level, and
neurocognitive development was measured using expert-assessment of
cognitive function and neuroimaging using magnetic resonance imaging
(MRI). Using a combination of classical statistical analysis and machine
learning, we showed that the development of the gut microbiome, the
brain, and children's cognitive abilities are intimately linked, with
both microbial taxa and gene functions able to predict cognitive
performance and brain structure.

\section*{Results}

\subsection*{Outline}

\begin{itemize}
  \item Intial description: Cohort size, demographics, data types
  \item Basic analysis: Summary statistics, ordinations, permanovas, mantel tests
  \item Exploration of taxonomic profiles + cogscores / brain, cross-sectional. Linear models. \hl{Is this necessary??}
  \item Try HAllA on multivariate data?
  \item Predictive model (RF) - taxonomic profiles + cogscores.
  \item Taxonomic profiles + brain structure
  \item Gene functions + metabolomes
\end{itemize}

\section*{Discussion}

The relationship between the gut microbiome and brain function via the
gut-microbiome-brain axis has gained increasing acceptance largely as a
result of human epidemiological studies investigating atypical
neurocognition (eg anxiety and depression, neurodegeneration, attention
deficit / hyperactivity disorder, and autism) and mechanistic studies in
animal models. The results from these studies point to the possibility
that gut microbes and their metabolism may be causally implicated in
cognitive development, but this study is the first to our knowledge that
directly investigates microbial species and their genes in relation to
typical development in young children. Understanding the gut-brain-axis
in early life is particularly important, since differences or
interventions in early life can have outsized and longer-term
consequences than those at later ages. Further, even in the absence of
causal impacts of microbial metabolism, identifying risk factors that
could point to other early interventions would also have value.

The use of shotgun metagenomic sequencing enabled us to get
species-level resolution of microbial taxa. A previous study of
cognition in 3 year old subjects used 16S rRNA gene amplicon sequencing,
and showed that genera from the Lachnospiraceae family as well as
unclassified Clostridiales (now Eubacteriales) were associated with
higher scores on the Ages and Stages Questionnaire
\href{https://www.zotero.org/google-docs/?SB5Lwa}{(Sordillo et al.,
2019)}. However, each of these clades encompass dozens of genera with
diverse functions, each of which may have different effects. Indeed,
several of the taxa that were positively associated with cognitive
function in this study, including \emph{B. wexlerae}, the only species
identified by linear models in children under 6 months, \emph{D.
longicatena}, \emph{R. faecis}, and \emph{A. finegoldii} are
Clostridiales, as is \emph{R. gnavus}, which we found was negatively
associated with cognitive function (Figure 2A-B). This kind of
species-level resolution is typically not possible with amplicon
sequencing.

We identified several species in the family Eggerthelaceae that were
associated with cognitive function, including \emph{Gordonibacter
pamelaeae, Aldercreutzia equolofaciens, Asaccharobacter celatus}
(formerly regarded a subspecies of \emph{A. equolofaciens}
\href{https://www.zotero.org/google-docs/?OEhguI}{(Takahashi et al.,
2021)}), and \emph{Eggerthella lenta}. Many members of this family are
known in part due to unique metabolic activities. For example, \emph{A.
equolofaciens} produces the nonsteroidal estrogen equol from isoflaven
found in soybeans
\href{https://www.zotero.org/google-docs/?I2qNyB}{(Wang et al., 2005)},
and \emph{G. pamelaeae} can metabolize the polyphenol ellagic acid
(found in pomegranates and some berries) into urolithin, which has been
shown in some studies to have a neuroprotective effect
\href{https://www.zotero.org/google-docs/?oTN8Lw}{(Gong et al., 2022;
Selma et al., 2014)}. \emph{E. lenta} has been extensively studied for
its ability to metabolize drug compounds such as the plant-derived heart
medication digoxin
\href{https://www.zotero.org/google-docs/?tH2Qeq}{(Haiser et al.,
2013)}. The metabolic versatility of this clade, and the large number of
species that are associated with cognition make these microbes prime
targets for further mechanistic studies.

In addition to improved species-level resolution, shotgun metagenomic
sequencing also enables gene-functional insight. We showed here that
genes for the metabolism of SCFAs, both their degradation and synthesis,
are associated with cognitive function scores. However, while the
differential abundance of genes for the metabolism of neuroactive
compounds like these is suggestive, it is difficult to reason about the
relationship between levels of these genes and the gut concentrations of
the molecules their product enzymes act on. For example, while it might
be intuitive to reason that increased levels of menaquinone synthesis
genes is indicative of increased menaquinone, it could be the case that
menaquinone deficiency selects for microbes that can synthesize it. For
the same reason, increased propionate degradation genes may
counterintuitively be indicative of high levels of propionate in the gut
lumen, since high propionate would select for microbes that can
metabolize it. For this reason, future studies coupling shotgun
metagenomics with stool metabolomics could improve our understanding of
the relationship between microbial metabolism and cognitive development.
Further, strain-level analysis linking specific gene content in species
of interest could further refine targeted efforts at identifying
specific metabolic signatures of microbe-brain interactions.

The use of multiple age-appropriate cognitive assessments that could be
normalized to a common scale enabled us to analyze microbial
associations across multiple developmental periods, but carries several
drawbacks. In particular, the test-retest reliability, as well as
systematic differences between test administrators may introduce
substantial noise into these observations, particularly in the youngest
children. In addition, our study period overlapped with the beginning of
the COVID-19 pandemic, and we and others have observed some reduction in
measured scores for children that were assessed after the implementation
of lockdowns. In our subject set for this study, these effects are more
pronounced in some age groups due to our sampling schedule
\href{https://www.zotero.org/google-docs/?L8GnRj}{(Blackwell et al.,
2022; Deoni et al., 2021)} (Supplementary Figure 3).

This analysis allowed us to establish links between microbial taxa and
their functional potential with cognition and brain structure. Although
we cannot test causality or the chemistry behind the interactions
between gut microbial taxa, gut, and brain, this study provides clear
and statistically significant associations between the infant and early
child gut microbiota and neurocognition. Future studies should focus on
characterizing the early-life microbiome and neurocognitive development
across different geographic regions and lifestyles such as covering
traditionally understudied low-resource urban, peri-urban and rural
communities to obtain the more comprehensive understanding of the
variability within the different gut microbiomes reflects on
neurocognition. These studies would also provide us with the wealth of
data on different strains from the same species to better understand the
effect of genes and their products. Furthermore, culturing and microbial
community enrichment studies combined with genetic manipulation and
genomic approaches to understand microbial metabolism at the molecular
level is the key, as the metabolic functions shape and influence the
human host and its health. The discovery of the neuroactive metabolites
could provide us with biomarkers for early detection or necessary
medicinally useful molecules that can be applied in intervention.


\section*{Material and methods}

\subsubsection{Study Ethics}

All procedures for this study were approved by the local institutional
review board at Rhode Island Hospital, and all experiments adhered to
the regulation of the review board. Written informed consent was
obtained from all parents or legal guardians of enrolled participants.

\subsubsection{Participants}

Data used in this study were drawn from the ongoing longitudinal
RESONANCE study of healthy and neurotypical brain and cognitive
development, based at Brown University in Providence, RI, USA. The
RESONANCE study is part of the NIH initiative Environmental influences
on Child Health Outcomes (ECHO)
\href{https://www.zotero.org/google-docs/?0tZanX}{(Forrest et al., 2018;
Gillman and Blaisdell, 2018)}, a longitudinal observational study of
healthy and neurotypical brain development that spans the fetal and
infant to adolescent life stages, combining neuroimaging (magnetic
resonance imaging, MRI), neurocognitive assessments, bio-specimen
analyses, subject genetics, environmental exposures such as lead, and
rich demographic, socioeconomic, family and medical history information.
From the RESONANCE cohort, 361 typically-developing children between the
ages of 2.8 months and 10 years old (median age 2 years, 2 months) were
selected for analysis in this study.

General participant demographics are provided in Table 1. Children are
representative of the RI population. Children in the RESONANCE cohort
were born full-term (\textless{} 37 weeks gestation) with height and
weight normal for gestational age, and from uncomplicated singleton
pregnancies. Children with known major risk factors for developmental
abnormalities at enrollment were excluded. In addition to screening at
the time of enrollment, on-going screening for worrisome behaviors using
validated tools was performed to identify at-risk children and remove
them from subsequent analysis.

Exclusion criteria included: \emph{in utero} exposure to alcohol,
cigarette or illicit substance exposure; preterm (\textless{} 37 weeks
gestation) birth; small for gestational age or less than 1500 g; fetal
ultrasound abnormalities; preeclampsia, high blood pressure, or
gestational diabetes; 5 minute APGAR scores \textless{} 8; NICU
admission; neurological disorder (e.g., head injury resulting in loss of
consciousness, epilepsy); and psychiatric or learning disorder
(including maternal depression) in the infant, parents, or siblings
requiring medication in the year prior to pregnancy.

Demographic and other non-biospecimen data such as race and ethnicity,
parental education and occupation, feeding behavior (breast- and
formula-feeding), child weight and height, were collected through
questionnaires or direct examination as appropriate. All data were
collected at every assessment visit, if possible.

\subsubsection{Cognitive Assessments}

Overall cognitive function was assessed using age-appropriate methods.
For children from birth to 30 months, we used an Early Learning
Composite as assessed via the Mullen Scales of Early Learning (MSEL)
\href{https://www.zotero.org/google-docs/?NSyykd}{(Mullen and others,
1995)}, a standardized and population-normed tool for assessing fine and
gross motor, expressive and receptive language, and visual reception
functioning in children from birth through 68 months of age. The
Wechsler Intelligence Quotient for Children (WISC)
\href{https://www.zotero.org/google-docs/?OopdA1}{(Wechsler, 2012)} is
an individually administered standard intelligence test for children
aged 6 to 16 years. It derives a full scale intelligence quotient (IQ)
score, which we used to assess overall cognitive functioning. The fourth
edition of the Wechsler Preschool and Primary Scale of Intelligence
(WPPSI-IV) is an individually administered standard intelligence test
for children aged 2 years 6 months to 7 years 7 months, trying to meet
the increasing need for the assessment of preschoolers. Just as the
WISC, it derives a full scale IQ score, which we used to assess overall
cognitive functioning.

\subsubsection{Stool Sample Collection an
Sequencing}

Stool samples (n=493) were collected by parents in OMR-200 tubes
(OMNIgene GUT, DNA Genotek, Ottawa, Ontario, Canada), stored on ice, and
brought within 24 hrs to the lab in RI where they were immediately
frozen at -80 ˚C. Stool samples were not collected if the subject had
taken antibiotics within the last two weeks. DNA extraction was
performed at Wellesley College (Wellesley, MA). Nucleic acids were
extracted from stool samples using the RNeasy PowerMicrobiome kit,
excluding the DNA degradation steps. Briefly, the samples were lysed by
bead beating using the Powerlyzer 24 Homogenizer (Qiagen, Germantown,
MD) at 2500 rpm for 45 s and then transferred to the QIAcube (Qiagen,
Germantown, MD) to complete the extraction protocol. Extracted DNA was
sequenced at the Integrated Microbiome Resource (IMR, Dalhousie
University, NS, Canada).

Shotgun metagenomic sequencing was performed on all samples. A pooled
library (max 96 samples per run) was prepared using the Illumina Nextera
Flex Kit for MiSeq and NextSeq from 1 ng of each sample. Samples were
then pooled onto a plate and sequenced on the Illumina NextSeq 550
platform using 150+150 bp paired-end ``high output'' chemistry,
generating 400 million raw reads and 120 Gb of sequence per plate.

\subsubsection{Machine learning for cognitiv
development}

Prediction of cognitive scores was carried out as a set of regression
experiments targeting real-valued continuous assessment scores.
Different experiment sets were designed to probe how different
representations of the gut microbiome (taxonomic profiles, Functional
profiles encoded as ECs) would behave, with and without the addition of
demographics (sex and maternal education as a proxy of socioeconomic
status) on participants from different age groups. Age (in months) was
provided as a covariate for all models (Table 3).

\textbf{Table 5. Experimental design and input composition for Random
Forest experiments}



Random Forests (RFs) \href{https://paperpile.com/c/dPbU4e/VDqU}{(Breiman
1996)} were selected as the prediction engine and processed using the
DecisionTree.jl
\href{https://www.zotero.org/google-docs/?dQLyDs}{(Sadeghi et al.,
2022)} implementation, inside the MLJ.jl
\href{https://www.zotero.org/google-docs/?y8ywGI}{(Blaom et al., 2020)}
Machine Learning framework. Independent RFs were trained for each
experiment, using a set of default regression hyperparameters from
Breiman and Cutler
\href{https://www.zotero.org/google-docs/?d3gqFh}{(Breiman, 2001)}, on a
repeated cross-validation approach with different RNG seeds. One hundred
repetitions of 3-fold CV with 10 different intra-fold RNG states each
were employed, for a total of 3000 experiments per input set.

After the training procedures, the root-mean-square error (RMSE) for
cognitive assessment scores and mean absolute proportional error (MAPE)
for the brain segmentation data, along with Pearson\textquotesingle s
correlation coefficient (R) were benchmarked on the validation and train
sets. MAPE was chosen as the metric for brain segments due to magnitude
differences between median volumes of each segment, which would hinder
interpretation of raw error values without additional reference.

To derive biological insight from the models, the covariate variable
importances for all the input features, measured by mean decrease in
impurity (MDI, or GINI importance), was also analyzed. Leveraging the
distribution of results from the extensive repeated cross validation
experiments, rather than electing a representative model or picking the
highest validation-set correlation, a measure of model fitness
(\textbf{Equation 1}) was designed to weight the importances from each
trained forest. The objective was to give more weight to those with
higher benchmarks on the validation sets (or higher generalizability),
while penalizing information from highly overfit models, drawing
inspiration from the approach used on another work employing repeated CV
on Random Forests with high-dimensional, low sample size microbiome
datasets \{***\}. The resulting fitness-weighted importances were used
to generate the values in Figure 3.

\(fitness =\)

\textbf{Equation 1.} Mathematical expression of the fitness measure used
to weight feature importances based on model benchmarks

\subsubsection{MRI / segmentation}

MRI data was acquired at a 3T Siemens Trio scanner with the following
parameters: TE=5.6msec, TR=1400msec, FA=15 degrees, 1.1x1.1x1.1mm
resolution, 160x160 matrix, with an average of 112 slices. FOV was
adjusted to infant size. Using a combination of linear and nonlinear
image registration, we created representative age-specific templates
using the ANTs package
\href{https://www.zotero.org/google-docs/?ZlWDfL}{(Avants et al.,
2014)}. After age specific templates were created, a single flow from
each age to the 12-month template was estimated and a final warp from
the 12-month template to standard adult MNI space was performed. For
each individual infant/ child brain image, we then calculated the warp
from their native T1w image space to their nearest-in-age template.
Using the resulting warps (native → nearest age template → 12-month
template → MNI template), we could move the standard adult brain atlas
to the space of an individual infant in a single step. In this case, we
used the Harvard-Oxford brain atlas to provide a coarse-grained
parcellation of individual brains into subcortical regions (e.g.,
thalamus, putamen) and total grey and white matter volumes (included as
part of the FSL package
\href{https://www.zotero.org/google-docs/?AapPN7}{(Jenkinson et al.,
2012)}. Total tissue and brainstem volume as well as left and right
hemisphere volumes were derived for total white and cortical gray
matter, lateral ventricle, thalamus, caudate, putamen, pallidum,
hippocampus, amygdala, and accumbens as well as total brainstem volume
\href{https://www.zotero.org/google-docs/?MCDDB9}{(Bruchhage et al.,
n.d.)}.

\subsubsection{Computational analysis of metagenomes}

Shotgun metagenomic sequences were analyzed using the bioBakery suite of
computational tools \href{https://paperpile.com/c/dPbU4e/vDau}{(Beghini
et al. 2021)}. First, KneadData (v0.7.7) was used to perform quality
control of raw sequence reads, such as read trimming and removal of
reads matching a human genome reference. Next, MetaPhlAn (v3.0.7, using
database mpa\_v30\_CHOCOPhlAn\_201901) was used to generate taxonomic
profiles by aligning reads to a reference database of marker genes.
Finally, HUMAnN (v3.0.0a4) was used to functionally profile the
metagenomes.

\subsubsection{Data and code availability}

Taxonomic and functional microbial profiles, as well as subject
demographics necessary for statistical analyses and machine learning are
available on the Open Science Framework
\href{https://www.zotero.org/google-docs/?61HLlM}{(Bonham et al.,
2022)}. Data processing, generation of summary statistics, and
generation of plots was performed using the julia programming language
\href{https://www.zotero.org/google-docs/?zj3HnY}{(Bezanson et al.,
2017; Bonham et al., 2021; Danisch and Krumbiegel, 2021)}. All code for
data analysis and figure generation, as well as scripts for automated
download of input files are available on github \{\{CITE zenodo\}\}.

\subsection{Research Standards}

\subsection{Acknowledgements}

Research funding: NIH UG3 OD023313 and Wellcome LEAP 1kD.

\end{document}
